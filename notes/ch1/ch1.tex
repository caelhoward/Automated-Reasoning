\documentclass[11pt]{article}

% Prefix for numedquestion's
\newcommand{\questiontype}{Question}


% Use this if your "written" questions are all under one section
% For example, if the homework handout has Section 5: Written Questions
% and all questions are 5.1, 5.2, 5.3, etc. set this to 5
% Use for 0 no prefix. Redefine as needed per-question.
\newcommand{\writtensection}{0}

\usepackage{amsmath, amsfonts, amsthm, amssymb}  % Some math symbols

\usepackage{centernot}
\usepackage{mathtools}

\setlength{\parindent}{0pt}

\begin{document}

\section{Propositional Logic}
\subsection{Syntax}

Logical Statements are formed using propositional variables $\{P_1,...,P_n\}$ also called Boolean Variables, propositional symbols, atomic propositions, and atoms.
\vspace{1em}

Sentences are formed accoring to the following rules:

\begin{itemize}
    \item Every propositional variable is a sentence
    \item If $\alpha$ and $\beta$ are sentences, then $\lnot \alpha, \alpha \land \beta, \text{and } \alpha \lor \beta$ are also sentences
\end{itemize}

These logical connectives define all logical statements, for example, $\alpha \implies \beta$ can be written as $\lnot \alpha \lor \beta$
\vspace{1em}

Positive and negative literals are represented by $P_i$ and $\lnot P_i$.
\vspace{3em}

\textbf{\Large{Propositional Knowledge Bases}}
\vspace{1em}

A \textbf{knowledge base} is a set of propositional sentences $\Delta  = \alpha_1,\alpha_2,...$. And represents the shorthand for the logical statement $\alpha_1 \land \alpha_2 ...$
\vspace{1em}

We define the set of propositional variables as $\Sigma = A,B,C...$ and we can define a knowledge base as follows
$$\Delta_1 = \lnot A \lor B$$
$$
\Delta_2 = \begin{cases}
\lnot A \lor B \implies C\\
\lnot (\lnot A \lor B) \implies \lnot C
\end{cases}
$$
A more complex knowledge base one can write involving the variables $\Sigma_3 = A,B,C,X,Y$:
$$\Delta_3 = \begin{cases}
    A \Rightarrow \lnot X\\
    \lnot A \Rightarrow X\\
    A \land B \Rightarrow Y\\
    \lnot (A \land B) \Rightarrow \lnot Y\\
    X \lor Y \Rightarrow C\\
    \lnot (X \lor Y) \Rightarrow \lnot C
    \end{cases}
$$
\vspace{1em}

The choice of propositional variables is quite important, and must be made in the context of given application

\subsection{Semantics}

The semantics of propositional logic defines logical properties of such sentences, including consistency and validity, and logical relationships among sentences, including implication and equivalence. Examples include:
\begin{itemize}
\item $A \land \lnot A$ is inconsistent
\item $A \lor \lnot A$ is valid
\item $A \text{ and } A \Rightarrow B \text{ imply } B$
\item $A \lor B$ is equivalent to $B \lor A$
\end{itemize}

\subsubsection{Truth at a world}

A world $\omega$ is a function that assigns a value true/false to each propositional variable $P_i$. It represents a particular state of affairs in which the value of each propositional variable is known. The semantics of propositional logic is based on a simple definition which tells us whether a sentence $\alpha$ is true at a particular world $\omega$ written as $\omega \models P_i$
\begin{itemize}
\item $\omega \models P_i \text{ iff } \omega(P_i) = true$
\item $\omega \models \lnot \alpha $ iff $\omega \not\models P_i$
\item $\omega \models \alpha \lor \beta$ iff $\omega \models \alpha$ or $\omega \models \beta$
\item $\omega \models \alpha \land \beta$ iff $\omega \models \alpha$ and $\omega \models \beta$
\end{itemize}
A world is called a truth assignment or an interpretaion

\subsubsection{Logical Properties}

We say $\alpha$ is consistent iff there is at least one world $\omega$ at which $\alpha$ is true, otherwise it is inconsistent. The propblem of deciding the satisfiability of a propositional sentence is the first problem proven to be NP complete.
\vspace{1em}

A set of $n$ propositional variables generally leads to a total of $2^n$ possible worlds.
\vspace{1em}

False is used to denote a sentence which is unsatisfiable. It is valid iff it is true at every world. An invalid sentence is simply one that is invalid is some world. A sentence is valid iff its negation is inconsistent.
\vspace{1em}

The symbol true is used to denote a sentence which is valid. It is common to write $\models \alpha$ in order to indicate that sentence $\alpha$ is valid.
\vspace{1em}

A sentence is complete iff it is true at exactly one world.

\subsubsection{Logical relationships}
A logical property applies to a single sentence, while a logical relationship applies to two or more sentences.

\begin{itemize}
\item Two sentences are equivalent iff they are true at the same set of worlds
\item Two sentences are mutually exclusive iff they are never true at the same world
\item Two sentences are exhaustive iff one of them is true at each world
\end{itemize}
\vspace{1em}

For implies, $\alpha \Rightarrow \beta$ iff $\omega \models \beta$ whenever $\omega \models \alpha$ for all worlds $\omega$. We can write $\alpha \models \beta$ to say that sentence $\alpha$ implies sentence $\beta$.
\vspace{1em}

The property $\alpha \models \beta$ and $\alpha \not\models \beta$ holds when the sentence $\alpha$ is complete.

\subsection{Knowledge as a set of possible worlds}
The set of worlds that satisfy a sentence $\alpha$ are called the models of $\alpha$ and denoted by $Mods(\alpha) = \{\omega : \omega \models \alpha\}$

\end{document}
